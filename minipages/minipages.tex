\documentclass{article}
\usepackage[a4paper,margin=30mm,landscape]{geometry}
\usepackage{xcolor}
\usepackage[colorlinks=true]{hyperref}

% Define Folder Shapes
\setlength{\unitlength}{0.5mm}
\newsavebox{\foldera}
\savebox{\foldera} (40,32)[bl]{% definition
    \multiput(0,0)(0,28){2}
    {\line(1,0){40}}
    \multiput(0,0)(40,0){2}
    {\line(0,1){28}}
    \put(1,28){\oval(2,2)[tl]}
    \put(1,29){\line(1,0){5}}
    \put(9,29){\oval(6,6)[tl]}
    \put(9,32){\line(1,0){8}}
    \put(17,29){\oval(6,6)[tr]}
    \put(20,29){\line(1,0){19}}
    \put(39,28){\oval(2,2)[tr]}
}

\newsavebox{\folderb}
\savebox{\folderb} (40,32)[l]{% definition
    \put(0,14){\line(1,0){8}}
    \put(8,0){\usebox{\foldera}}
}
    
\def\folders{%
    \begin{picture}(120,168)
        \put(34,26){\line(0,1){102}}
        \put(14,128){\usebox{\foldera}}
        \multiput(34,86)(0,-37){3}
        {\usebox{\folderb}}
    \end{picture}%
}

\def\arrows{%
    \setlength{\unitlength}{0.75mm}
    \begin{picture}(60,40)
        \put(30,20){\vector(1,0){30}}
        \put(30,20){\vector(4,1){20}}
        \put(30,20){\vector(3,1){25}}
        \put(30,20){\vector(2,1){30}}
        \put(30,20){\vector(1,2){10}}
        \thicklines
        \put(30,20){\vector(-4,1){30}}
        \put(30,20){\vector(-1,4){5}}
        \thinlines
        \put(30,20){\vector(-1,-1){5}}
        \put(30,20){\vector(-1,-4){5}}
    \end{picture}%
}
\usepackage{lipsum}

\title{The Dark Sorcery of Minipages}
\author{Raphael Frey\quad
\href{https://github.com/alpenwasser/TeX/tree/master/minipages}
     {\nolinkurl{https://github.com/alpenwasser/TeX/}}}

\date{\today}
% **************************************************************************** %
\begin{document}
% **************************************************************************** %

\maketitle


For understanding how minipages behave, it  is helpful to have some additional
text in a document, such as  this. The tricky thing about minipages is usually
to get them to align nicely in  the vertical dimension. So, let us create some
minipages, and see what they do.
%An optional  first argument of  t (for  top) or b  (bottom) aligns the  top or
%bottom line  of the parbox  with the  text line. More precisely,  the optional
%argument causes the  reference point of the  box to be the  reference point of
%the top or bottom line of its contents.

\newlength{\parindentBackup}
\setlength{\parindentBackup}{\parindent}
\fbox{%
    \begin{minipage}{0.45\textwidth}
        We are now inside a minipage which was created by
        \texttt{\textbackslash{}begin\{minipage\}\{0.45\textbackslash{}textwidth\}}.

        As you perhaps suspect, the content  of the two minipages on this page
        is  vertically centered  with respect  to the  minipages.

        Also        note        that       inside        minipages,        the
        \texttt{\textbackslash{}parindent}  dimention   is  set  to   zero  by
        default. You can still set it to anything you like by using the usual
        \texttt{\textbackslash{}setlength\{\textbackslash{}parindent\}\{<length>\}}
        syntax\dots

        \setlength{\parindent}{\parindentBackup}
        \ldots as was done right  before starting this paragraph. In fact, the
        \texttt{\textbackslash{}parindent}  length  is  now (i.e.  right  from
        the  start of  this paragraph)  set  to its  original value. This  was
        accomplished by first creating a new length:
        \texttt{\textbackslash{}newlength\{\textbackslash{}parindentBackup\}}, 
        and then  storing the  value of  \texttt{\textbackslash{}parindent} in
        that length \emph{before} the minipage was opened, via:
        \texttt{\textbackslash{}setlength\{\textbackslash{}parindentBackup\}\{\textbackslash{}parindent\}}
        Then \texttt{\textbackslash{}parindent}  was restored to  its original
        value before the start of this paragraph.

        Lastly,  you may  have noticed  that the  minipage itself  is indented
        with  respect  to  regular  text  on  the  page. That  is  because  it
        is  basically  the  starting  element  of a  paragraph,  and  as  such
        \texttt{\textbackslash{}parindent}  was  applied   to  it  before  the
        minipage was started. We will fix this on the next page.
    \end{minipage}}
\hspace{0.1\textwidth}
\fbox{%
    \noindent\begin{minipage}{0.33\textwidth}
        \arrows
        \footnote{%
            Source              for             picture              examples:
            \href{https://en.wikibooks.org/wiki/LaTeX/Picture}
                 {https://en.wikibooks.org/wiki/LaTeX/Picture}%
        }
        \footnote{%
            Footnotes   inside    minipages   can    be   created    via   the
            \texttt{\textbackslash{}footnote} command.%
        }
        \footnote{%
            Note  the two  different  counters for  the  footnotes inside  the
            minipage and outside. This  can be at least  somewhat adjusted, if
            so desired, see for example here:
            \href{http://tex.stackexchange.com/questions/18499/how-to-change-symbol-for-footnote-in-minipage}
                 {http://tex.stackexchange.com/questions/18499/how-to-change-symbol-for-footnote-in-minipage}%
        }
        \footnotemark[1]
        \footnotemark[2]
    \end{minipage}}

\footnotetext[1]{%
    We       can       also        put       footnotes       outside       the
    minipage     via      the     \texttt{\textbackslash{}footnotemark}     --
    \texttt{\textbackslash{}footnotetext}    syntax,   optionally    with   an
    additional  numeric  index:  \texttt{\textbackslash{}footnotemark[1]}  and
    \texttt{\textbackslash{}footnotetext[1]\{text\}}. Using this syntax inside
    the  minipages   has  not  produced   reliable  results  for   the  author
    though. Beware of dragons. %
}


\footnotetext[2]{%
    Also note that hyperlinks to footnotes outside the minipage are broken.%
}

\newpage
\textcolor{gray}{\lipsum[1]}
\noindent\fbox{%
    \begin{minipage}{0.33\textwidth}
        As  can  be seen  when  comparing  the  alignment of  this  minipage's
        content  with  the  dummy  text  right  above,  the  minipage  is  now
        not  indented   with  respect  to  the   rest  of  the  text   of  the
        page. This  was  achieved  by  preceding  the  minipage  command  with
        \texttt{\textbackslash{}noindent}. This will  also work for  any other
        paragraph, should you so wish (though usually it is not recommended).

        Anyway, enough  about indentation: Maybe  you think that  aligning the
        picture on the right in reference  to the vertical center lines of the
        minipages  is not  what you  want. Maybe you  want the  picture to  be
        aligned with the top of the first minipage?

        Easier said than done! But alright: First things first. Have a look at
        the minipage  on the right. Notice  anything? The picture seems  to be
        more  aligned like  a letter  than  a picture,  or not? The  picture's
        bottom is on the same height as the bottom of the letters on the first
        line. That actually  has little to  do with the  minipage environment,
        but it  is important to know  for getting things aligned  as we (okay,
        \emph{I}) want later on.

        Now, if I may direct your attention  to that red reference line in the
        middle  of the  page. The  reference line  is  outside the  minipages,
        meaning it is  part of the regular  text flow. Note that it  is in the
        vertical center of the two minipages.
    \end{minipage}}
\hspace{0.05\textwidth}
\textcolor{red}{This is the reference line!}
\hfill
\noindent\fbox{%
    \begin{minipage}{0.33\textwidth}
        \arrows
        \lipsum[2]
    \end{minipage}}

\noindent\fbox{%
    \begin{minipage}[t]{0.33\textwidth}
        This is  what happens  when you create  a minipage  environment (well,
        two, to be precise) with the optional argument for top alignment, like
        so:
        \texttt{\textbackslash{}begin\{minipage\}[t]\{0.33\textbackslash{}textwidth\}}

        Note that  each minipage's  first line is  aligned with  the reference
        line now  (to be more accurate:  the baselines of the  two first lines
        in  the minipages  are  aligned  with the  baseline  of the  reference
        line). However,  because  the picture's  bottom  is  aligned with  the
        baseline of the  minipage's first line, and thus with  the baseline of
        the  reference  line, this  is  not  exactly  what  one might  call  a
        desirable outcome.

        We  can  also  call  a  minipage with  the  \texttt{[c]}  (for  center
        alignment, which is also the default  which gets called when you don't
        specify an  optional alignment parameter) or  \texttt{[b]} (for bottom
        alignment).
    \end{minipage}}
\hspace{0.05\textwidth}
{This is the reference line!
\hfill
\fbox{%
    \begin{minipage}[t]{0.33\textwidth}
        \arrows
        \lipsum[2]
    \end{minipage}}

\noindent\fbox{%
    \begin{minipage}[t]{0.33\textwidth}
        This   can  actually   be   used  for   some   rather  odd   placement
        configurations, like on this page, where the left minipage was created
        by calling
        \texttt{\textbackslash{}begin\{minipage\}[t]\{0.33\textbackslash{}textwidth\}}
        for top alignment, and the right one was created by
        \texttt{\textbackslash{}begin\{minipage\}[b]\{0.33\textbackslash{}textwidth\}}
        for bottom alignment.
    \end{minipage}}
\hspace{0.05\textwidth}
This is the reference line!
\hfill
\fbox{%
    \begin{minipage}[b]{0.33\textwidth}
        \arrows
        \lipsum[2]
    \end{minipage}}

\noindent\fbox{%
    \begin{minipage}[t][][b]{0.33\textwidth}
        But enough about  that. What about getting that picture to  align in a
        sane manner?  That  requires some more optional  arguments. To be more
        precise:

        \texttt{\textbackslash{}begin\{minipage\}[t][][b]\{0.33\textbackslash{}textwidth\}}

        Okay,   don't   panic. The   first  argument   \texttt{[t]}   is   the
        top-alignment one  from before. Nothing  new there. The middle  one is
        the minipage's height. Note  that while it is empty in  this case (and
        in  most  other cases,  because  usually  we  don't care  to  manually
        calculate how high our  minipage is going to be), it  must be there if
        you want  to have  the last  option, \texttt{[b]}. That  last optional
        argument tells  the minipage to align  the top of the  first line with
        the baseline of the reference line.

        Personally,   I   still   find   this   to   be   a   bit   off   from
        perfection. Ideally, the top of the picture should be aligned with the
        top of the reference line, in my humble opinion.
\end{minipage}}
\hspace{0.05\textwidth}
This is the reference line!
\hfill
\fbox{%
    \begin{minipage}[t][][b]{0.33\textwidth}
        \begin{center}
        \arrows
        \end{center}
        \lipsum[2]
\end{minipage}}

\newpage
\noindent\rule{237mm}{0.5pt}

\vspace{-0.21em} % aligned to 1em negative vspate for the figure
%\vspace{-0.75em} % aligned to 1ex negative vspace for the figure
\noindent\fbox{%
    \begin{minipage}[t][][t]{0.33\textwidth}
        So let's try that then. First, we create the left minipage with

        \texttt{\textbackslash{}begin\{minipage\}[t][][t]\{0.33\textbackslash{}textwidth\}}

        Thus aligning its top line perfectly to the reference line.

        The   second  minipage's   content  (not   the  minipage   itself,  as
        you  can   see  by  the  frame)   is  shifted  by  negative   1em  via
        \texttt{\textbackslash{}vspace\{-1em\}} by placing that command inside
        the  minipage (see  the  example without  the frameboxes  below). This
        means that its content  slightly protrudes above the minipage. Whether
        or not you care about that is up to you.
        
        Note  that the  top of  the picture  is aligned  with the  top of  the
        reference \emph{line},  not its letters. In  the end, how you  wish to
        align this is up to you.
        
        Further down,  you can see a  version of this page  without the frames
        and the horizontal line, to get an  impression on how it would look in
        an actual document.
    \end{minipage}}
\hspace{0.05\textwidth}
This is the reference line!
\hfill
\fbox{%
    \begin{minipage}[t][][b]{0.33\textwidth}
        %\vspace{-1ex}
        \vspace{-1em}
        \begin{center}
        \arrows
        \end{center}
        \lipsum[2]
    \end{minipage}}


\newpage
\noindent\rule{237mm}{0.5pt}

\vspace{-0.45em} % aligned to 1em negative vspate for the figure

\noindent\fbox{\begin{minipage}[t][][t]{0.33\textwidth}
    Alternatively  to  inserting  a  negative   vshift,  we  can  also  use  a
    \texttt{\textbackslash{}raisebox}  command to  raise  the entire  minipage
    upwards.

    Note that  the top  of the  left minibox is  aligned with  the top  of the
    drawing, but not with the top of the right minipage.

    If we leave out the rules and frameboxes, we can see on the next two pages
    that the  result for the two  approaches, when leaving out  the frames and
    the horizontal line, is the same.
\end{minipage}}}
\hspace{0.05\textwidth}
This is the reference line!
\hfill
\fbox{\raisebox{1em}{\begin{minipage}[t][][b]{0.33\textwidth}
    \begin{center}
    \arrows
    \end{center}
    \lipsum[2]
\end{minipage}}}

\newpage

\noindent\begin{minipage}[t][][t]{0.33\textwidth}
    As  promised,  the   same  alignment  options,  but   without  the  visual
    clutter.

    The right minipage on this page is created by:
    \begin{verbatim}
\begin{minipage}[t][][b]{0.33\textwidth}
    \vspace{-1em}
    \begin{center}
        \arrows
    \end{center}
    \lipsum[2]
\end{minipage}
    \end{verbatim}
\end{minipage}
\hspace{0.05\textwidth}
This is the reference line!
\hfill
\begin{minipage}[t][][b]{0.33\textwidth}
    %\vspace{-1ex}
    \vspace{-1em}
    \begin{center}
    \arrows
    \end{center}
    \lipsum[2]
\end{minipage}

\newpage

\noindent\begin{minipage}[t][][t]{0.33\textwidth}
    And the  same thing, but with  a \texttt{\textbackslash{}raisebox} instead
    of a \texttt{vshift}.

    The right minipage in this case works as follows:

    \begin{verbatim}
\raisebox{1em}{%
    \begin{minipage}[t][][b]{0.33\textwidth}
        \begin{center}
            \arrows
        \end{center}
    \lipsum[2]
\end{minipage}}
    \end{verbatim}

    If  you  overlay   this  page  and  the  previous  one,   they  should  be
    visually  indistinguishable  (well,  at   least  the  reference  line  and
    the  right minipage;  obviously  the  text in  the  left  minipage is  not
    identical). So which  approach you prefer is  up to you. The usage  of the
    \texttt{\textbackslash{}raisebox} command  is the cleaner approach  in the
    author's opinion,  but whether  the reader holds  the author's  opinion in
    sufficiently  high  regard to  agree  with  that  argument  is yet  to  be
    determined.
\end{minipage}
\hspace{0.05\textwidth}
This is the reference line!
\hfill
\raisebox{1em}{\begin{minipage}[t][][b]{0.33\textwidth}
    \begin{center}
    \arrows
    \end{center}
    \lipsum[2]
\end{minipage}}

\newpage
\noindent\fbox{%
    \begin{minipage}[t][][t]{0.33\textwidth}
        If the  minipages on both  sides open with  text, the command  as used
        earlier produces  the more pleasing  result I'd say, by  aligning both
        top lines  of the minipages  with the reference line. The  commands as
        used on this page are:

        \texttt{\textbackslash{}begin\{minipage\}[t][][t]\{0.33\textbackslash{}textwidth\}}
\end{minipage}}
\hspace{0.05\textwidth}
This is the reference line!
\hfill
\noindent\fbox{%
    \begin{minipage}[t][][t]{0.33\textwidth}
        \lipsum[4]
        \begin{center}
        \arrows
        \end{center}
        \lipsum[4]
\end{minipage}}

\newpage
\noindent\fbox{%
    \begin{minipage}[t][][t]{0.32\textwidth}
        \lipsum[1-6]
    \end{minipage}}
\hspace{0.01\textwidth}
\noindent\begin{minipage}[t][][t]{0.32\textwidth}
        A  minipage will  never be  broken  across different  pages, and  will
        therefore happily  flow into page  margins in both the  horizontal and
        vertical direction, or even wander right off the page itself.
\end{minipage}
\hspace{0.01\textwidth}
\noindent\fbox{%
    \begin{minipage}[t][][t]{0.52\textwidth}
        \lipsum[1-6]
    \end{minipage}}

\newpage

In theory, minipages can be used at arbitrary places inside text, like on this
page.
\fbox{\begin{minipage}{0.1\textwidth}
    This minipage is center-positioned, without any optional arguments.
    \end{minipage}}
Whether or not this  yields a desirable outcome is left as  an exercise to the
reader.  Placing minipages randomly inside  flowing text can yield some rather
odd results, after all, as can easily be seen here.
\fbox{\begin{minipage}[t]{0.1\textwidth}
    This minipage is top-positioned, via a single \texttt{[t]} argument.
    \end{minipage}}
Now let us  continue to write some  random text in order to  create some words
with which to fill this page so that we can see more minipage behavior.
\fbox{\begin{minipage}[b]{0.15\textwidth}
    This   minipage   is   bottom-positioned  by   the   single   \texttt{[b]}
    argument. Also, it has more text in it than the two previous ones, so that
    it grows  a bit taller. But  it's also a bit  wider, because why  the hell
    not.
    \end{minipage}}
\fbox{\begin{minipage}[t][][b]{0.1\textwidth}
    And let's  make another one right  next to it, with  the \texttt{[t][][b]}
    syntax.
    \end{minipage}}
Though it might show  up on a different place due to line  breaks and all that
stuff.
\fbox{\begin{minipage}[t][][t]{0.1\textwidth}
    This minipage has been positioned via \texttt{[t][][t]}.
    \end{minipage}}
The difference should not be too difficult to spot.

% **************************************************************************** %
\end{document}
% **************************************************************************** %
