\documentclass{article}
\usepackage[a4paper,margin=30mm,landscape]{geometry}
\usepackage{xcolor}

% Define Folder Shapes
\setlength{\unitlength}{0.5mm}
\newsavebox{\foldera}
\savebox{\foldera} (40,32)[bl]{% definition
    \multiput(0,0)(0,28){2}
    {\line(1,0){40}}
    \multiput(0,0)(40,0){2}
    {\line(0,1){28}}
    \put(1,28){\oval(2,2)[tl]}
    \put(1,29){\line(1,0){5}}
    \put(9,29){\oval(6,6)[tl]}
    \put(9,32){\line(1,0){8}}
    \put(17,29){\oval(6,6)[tr]}
    \put(20,29){\line(1,0){19}}
    \put(39,28){\oval(2,2)[tr]}
}

\newsavebox{\folderb}
\savebox{\folderb} (40,32)[l]{% definition
    \put(0,14){\line(1,0){8}}
    \put(8,0){\usebox{\foldera}}
}
    
\def\folders{%
    \begin{picture}(120,168)
        \put(34,26){\line(0,1){102}}
        \put(14,128){\usebox{\foldera}}
        \multiput(34,86)(0,-37){3}
        {\usebox{\folderb}}
    \end{picture}%
}

\def\arrows{%
    \setlength{\unitlength}{0.75mm}
    \begin{picture}(60,40)
        \put(30,20){\vector(1,0){30}}
        \put(30,20){\vector(4,1){20}}
        \put(30,20){\vector(3,1){25}}
        \put(30,20){\vector(2,1){30}}
        \put(30,20){\vector(1,2){10}}
        \thicklines
        \put(30,20){\vector(-4,1){30}}
        \put(30,20){\vector(-1,4){5}}
        \thinlines
        \put(30,20){\vector(-1,-1){5}}
        \put(30,20){\vector(-1,-4){5}}
    \end{picture}%
}
\usepackage{lipsum}

% **************************************************************************** %
\begin{document}
% **************************************************************************** %


Source for picture examples: \texttt{https://en.wikibooks.org/wiki/LaTeX/Picture}

For understanding how minipages behave, it  is helpful to have some additional
text in a document, such as  this. The tricky thing about minipages is usually
to get them to align nicely in  the vertical dimension. So, let us create some
minipages, and see what they do.
%An optional  first argument of  t (for  top) or b  (bottom) aligns the  top or
%bottom line  of the parbox  with the  text line. More precisely,  the optional
%argument causes the  reference point of the  box to be the  reference point of
%the top or bottom line of its contents.

\newlength{\parindentBackup}
\setlength{\parindentBackup}{\parindent}
\fbox{%
    \begin{minipage}{0.45\textwidth}
        We are now inside a minipage which was created by
        \texttt{\textbackslash{}begin\{minipage\}\{0.33\textbackslash{}textwidth\}}.

        As you perhaps suspect, the content  of the two minipages on this page
        is  vertically centered  with respect  to the  minipages.

        Also        note        that       inside        minipages,        the
        \texttt{\textbackslash{}parindent}  dimention   is  set  to   zero  by
        default. You can still set it to anything you like by using the usual
        \texttt{\textbackslash{}setlength\{\textbackslash{}parindent\}\{<length>\}}
        syntax\dots

        \setlength{\parindent}{\parindentBackup}
        \ldots as was done right  before starting this paragraph. In fact, the
        \texttt{\textbackslash{}parindent}  length  is  now (i.e.  right  from
        the  start of  this paragraph)  set  to its  original value. This  was
        accomplished by first creating a new length:
        \texttt{\textbackslash{}newlength\{\textbackslash{}parindentBackup\}}, 
        and then  storing the  value of  \texttt{\textbackslash{}parindent} in
        that length \emph{before} the minipage was opened, via:
        \texttt{\textbackslash{}setlength\{\textbackslash{}parindentBackup\}\{\textbackslash{}parindent\}}
        Then \texttt{\textbackslash{}parindent}  was restored to  its original
        value before the start of this paragraph.

        Lastly,  you may  have noticed  that the  minipage itself  is indented
        with  respect  to  regular  text  on  the  page. That  is  because  it
        is  basically  the  starting  element  of a  paragraph,  and  as  such
        \texttt{\textbackslash{}parindent}  was  applied   to  it  before  the
        minipage was started. We will fix this on the next page.
    \end{minipage}}
\hspace{0.1\textwidth}
\fbox{%
    \noindent\begin{minipage}{0.33\textwidth}
        \arrows
    \end{minipage}}

\fbox{%
    \noindent\begin{minipage}{0.33\textwidth}
        As  can  be seen  when  comparing  the  alignment of  this  minipage's
        content  with  the  dummy  text  right  above,  the  minipage  is  now
        not  indented   with  respect  to  the   rest  of  the  text   of  the
        page. This  was  achieved  by  preceding  the  minipage  command  with
        \texttt{\textbackslash{}noindent}. This will  also work for  any other
        paragraph, should you so wish (though usually it is not recommended).

        Anyway, enough  about indentation: Maybe  you think that  aligning the
        picture on the right in reference  to the vertical center lines of the
        minipages  is not  what you  want. Maybe you  want the  picture to  be
        aligned with the top of the first minipage?

        Easier said than done! But alright: First things first. Have a look at
        the minipage  on the right. Notice  anything? The picture seems  to be
        more  aligned like  a letter  than  a picture,  or not? The  picture's
        bottom is on the same height as the bottom of the letters on the first
        line. That actually  has little to  do with the  minipage environment,
        but it  is important to know  for getting things aligned  as we (okay,
        <em>I</em>) want later on.

        Now, if I may direct your attention  to that red reference line in the
        middle  of the  page. The  reference line  is  outside the  minipages,
        meaning it is  part of the regular  text flow. Note that it  is in the
        vertical center of the two minipages.
    \end{minipage}}
\hspace{0.05\textwidth}
\textcolor{red}{This is the reference line!}
\hfill
\fbox{%
    \noindent\begin{minipage}{0.33\textwidth}
        \arrows
        \lipsum[2]
    \end{minipage}}

\fbox{%
    \noindent\begin{minipage}[t]{0.33\textwidth}
        This is  what happens  when you create  a minipage  environment (well,
        two, to be precise) with the optional argument for top alignment, like
        so:
        \texttt{\textbackslash{}begin\{minipage\}[t]\{0.33\textbackslash{}textwidth\}}

        Note that  each minipage's  first line is  aligned with  the reference
        line now. However,  because the picture's  bottom is aligned  with the
        bottom of the  minipage's first line, and thus with  the bottom of the
        reference line, this is not exactly what I'd call a desirable outcome.

        We  can  also  call  a  minipage with  the  \texttt{[c]}  (for  center
        alignment, which is also the default  which gets called when you don't
        specify an  optional alignment parameter) or  \texttt{[b]} (for bottom
        alignment).
    \end{minipage}}
\hspace{0.05\textwidth}
{This is the reference line!
\hfill
\fbox{%
    \begin{minipage}[t]{0.33\textwidth}
        \arrows
        \lipsum[2]
    \end{minipage}}

\fbox{%
    \noindent\begin{minipage}[t]{0.33\textwidth}
        This   can  actually   be   used  for   some   rather  odd   placement
        configurations, like on this page, where the left minipage was created
        by calling
        \texttt{\textbackslash{}begin\{minipage\}[t]\{0.33\textbackslash{}textwidth\}}
        for top alignment, and the right one was created by
        \texttt{\textbackslash{}begin\{minipage\}[b]\{0.33\textbackslash{}textwidth\}}
        for bottom alignment.
    \end{minipage}}
\hspace{0.05\textwidth}
This is the reference line!
\hfill
\fbox{%
    \begin{minipage}[b]{0.33\textwidth}
        \arrows
        \lipsum[2]
    \end{minipage}}

\noindent\fbox{%
    \begin{minipage}[t][][b]{0.33\textwidth}
        But enough about  that. What about getting that picture to  align in a
        sane manner?  That  requires some more optional  arguments. To be more
        precise:

        \texttt{\textbackslash{}begin\{minipage\}[t][][b]\{0.33\textbackslash{}textwidth\}}

        Okay,   don't   panic. The   first  argument   \texttt{[t]}   is   the
        top-alignment one  from before. Nothing  new there. The middle  one is
        the minipage's height. Note  that while it is empty in  this case (and
        in  most  other cases,  because  usually  we  don't care  to  manually
        calculate how high our minipage is going  to be), but it must be there
        if you want to have  the last option, \texttt{[b]}. That last optional
        argument tells  the minipage to align  the top of the  first line with
        the bottom of the baseline.

        Personally,   I   still   find   this   to   be   a   bit   off   from
        perfection. Ideally, the top of the picture should be aligned with the
        top of the reference line, in my humble opinion.
\end{minipage}}
\hspace{0.05\textwidth}
This is the reference line!
\hfill
\fbox{%
    \begin{minipage}[t][][b]{0.33\textwidth}
        \begin{center}
        \arrows
        \end{center}
        \lipsum[2]
\end{minipage}}

\newpage
\noindent\rule{237mm}{0.5pt}

\vspace{-0.21em} % aligned to 1em negative vspate for the figure
%\vspace{-0.75em} % aligned to 1ex negative vspace for the figure
\noindent\fbox{%
    \begin{minipage}[t][][t]{0.33\textwidth}
        So let's try that then. First, we create the left minipage with

        \texttt{\textbackslash{}begin\{minipage\}[t][][t]\{0.33\textbackslash{}textwidth\}}

        Thus aligning its top line perfectly to the reference line.

        The   second  minipage's   content  (not   the  minipage   itself,  as
        you  can   see  by  the  frame)   is  shifted  by  negative   1em  via
        \texttt{\textbackslash{}vspace\{-1em\}}. This  means that  its content
        slightly protrudes above  the minipage. Whether or not  you care about
        that is up to you.
        
        Note  that the  top of  the picture  is aligned  with the  top of  the
        reference \emph{line},  not its letters. In  the end, how you  wish to
        align this is up to you.
        
        The next page is  the same as this one, but without  all the lines, so
        that it  is easier to  get an  impression of how  it would look  in an
        actual document.
    \end{minipage}}
\hspace{0.05\textwidth}
This is the reference line!
\hfill
\fbox{%
    \begin{minipage}[t][][b]{0.33\textwidth}
        %\vspace{-1ex}
        \vspace{-1em}
        \begin{center}
        \arrows
        \end{center}
        \lipsum[2]
    \end{minipage}}

\newpage

\begin{minipage}[t][][t]{0.33\textwidth}
    And as promised, these are the same alignment options as on the previous
    page, but without the visual clutter. Created, again, by using

    \texttt{\textbackslash{}begin\{minipage\}[t][][t]\{0.33\textbackslash{}textwidth\}}

    and  by shifting  the right  minipage's content  upwards by  1em with  the
    \texttt{\textbackslash{}vspace\{-1em\}} command.

    I did  try out shifting  the entire  minipage upwards, but  somehow \LaTeX
    refused  to cooperate. Maybe  somebody more  versed than  me in  the inner
    workings of \LaTeX can figure that one out.
\end{minipage}
\hspace{0.05\textwidth}
This is the reference line!
\hfill
\begin{minipage}[t][][b]{0.33\textwidth}
    %\vspace{-1ex}
    \vspace{-1em}
    \begin{center}
    \arrows
    \end{center}
    \lipsum[2]
\end{minipage}

\newpage
\noindent\fbox{%
    \begin{minipage}[t][][t]{0.33\textwidth}
        If the  minipages on both  sides open with  text, the command  as used
        earlier produces  the more pleasing  result I'd say, by  aligning both
        top lines  of the minipages  with the reference line. The  commands as
        used on this page are:

        \texttt{\textbackslash{}begin\{minipage\}[t][][t]\{0.33\textbackslash{}textwidth\}}
\end{minipage}}
\hspace{0.05\textwidth}
This is the reference line!
\hfill
\noindent\fbox{%
    \begin{minipage}[t][][t]{0.33\textwidth}
        \lipsum[4]
        \begin{center}
        \arrows
        \end{center}
        \lipsum[4]
\end{minipage}}

% **************************************************************************** %
\end{document}
% **************************************************************************** %
