\documentclass[a4paper,oneside,11pt]{article}

\usepackage{lipsum}
\usepackage{booktabs}
\usepackage{xcolor}
\usepackage{framed}
\usepackage[%
        bookmarksnumbered=true,
        colorlinks=true,
        linkcolor=cyan!50!blue,
        citecolor=violet,
        urlcolor=purple,
    ]{hyperref}
\usepackage{hologo}
\usepackage[left=30mm,right=30mm,top=30mm,bottom=40mm]{geometry}
\usepackage[light,nott]{kpfonts}

\usepackage[T1]{fontenc}
\usepackage[utf8]{inputenc}

\newcommand\compar[2]{\texttt{\textbackslash #1\{#2\}}}
\newcommand\comm[1]{\texttt{\textbackslash#1}}
\newcommand\code[1]{\texttt{#1}}
\newcommand\XeLaTeX{\hologo{XeLaTeX}}
\newenvironment{framedcompact}
    {\begin{framed}\vspace{-1em}}
    {\vspace{-1em}\end{framed}}
\definecolor{TFFrameColor}{HTML}{4393c3}
\definecolor{TFTitleColor}{HTML}{FFFFFF}

\title{Changing Fonts in \LaTeX}
\author{Raphael Frey  <\texttt{webmaster@alpenwasser.net}>}


% ************************************************************************** %
\begin{document}
% ************************************************************************** %

\maketitle

\tableofcontents

\vspace{8em}

\begin{abstract}
    This document  presents a  very brief  overview of the  topic of  fonts in
    \LaTeX. A  quick introduction  to some  of the  terminology is  given, and
    various ways  are shown to change  fonts and different aspects  of them to
    one's liking.

    This document is in  no way, shape or form exhaustive;  the topic of fonts
    can fill  (and has filled)  entire books. But  it should suffice  for most
    day-to-day usage.
\end{abstract}


% -------------------------------------------------------------------------- %
\newpage
\section{Incomplete Summary For the Impatient}
\label{sec:summary}
% -------------------------------------------------------------------------- %

\begin{titled-frame}
{\textsf{Switching to Default Document Font}}
\vspace{-1em}
\begin{verbatim}
\normalfont The quick brown fox jumps over the lazy dog.
\end{verbatim}
\normalfont The quick brown fox jumps over the lazy dog.
\end{titled-frame}

\begin{titled-frame}
{\textsf{Switching Between Roman, Sans-Serif and Monospace Fonts}}
\vspace{-1em}
\begin{verbatim}
\textrm{The quick brown fox jumps over the lazy dog.}
\textsf{The quick brown fox jumps over the lazy dog.}
\texttt{The quick brown fox jumps over the lazy dog.}
\end{verbatim}
\textrm{The quick brown fox jumps over the lazy dog.}\\
\textsf{The quick brown fox jumps over the lazy dog.}\\
\texttt{The quick brown fox jumps over the lazy dog.}
\end{titled-frame}

\begin{titled-frame}
{\textsf{Font Weights}}
\vspace{-1em}
\begin{verbatim}
\textbf{The quick brown fox jumps over the lazy dog.}
\textmd{The quick brown fox jumps over the lazy dog.}
\end{verbatim}
\textbf{The quick brown fox jumps over the lazy dog.}\\
\textmd{The quick brown fox jumps over the lazy dog.}
\end{titled-frame}

\begin{titled-frame}
{\textsf{Font Shapes}}
\vspace{-1em}
\begin{verbatim}
\emph{The quick brown fox jumps over the lazy dog.}
\textit{The quick brown fox jumps over the lazy dog.}
\textsl{The quick brown fox jumps over the lazy dog.}
\textsc{The quick brown fox jumps over the lazy dog.}
\textup{The quick brown fox jumps over the lazy dog.}
\end{verbatim}
\emph{The quick brown fox jumps over the lazy dog.}\\
\textit{The quick brown fox jumps over the lazy dog.}\\
\textsl{The quick brown fox jumps over the lazy dog.}\\
\textsc{The quick brown fox jumps over the lazy dog.}\\
\textup{The quick brown fox jumps over the lazy dog.}
\end{titled-frame}


% -------------------------------------------------------------------------- %
\newpage
\section{Terminology: Font Families, Fonts and Glyphs}
\label{sec:font-families}
% -------------------------------------------------------------------------- %

A \emph{font family},  also referred to as  a \emph{typeface}, is a  font or a
set  of fonts  with common  features and  characteristics, comprising  all the
needed \emph{glyphs}. A glyph, in this  context, is a graphical representation
of a character,  which can be a  letter, a logogram, a  mathematical symbol or
other things\footnotemark.

\footnotetext{%
    Obviously typography is a vast and complex topic and there is no way I can
    do it  true justice  here. Even if  I wanted  to, I  am most  certainly no
    scholar  on  the  topic. But  for  the  purposes  of  this  document  this
    information shall suffice. The curious reader  shall feel free to research
    the topic in more depth, if so desired.}

For example, the following are all different glyphs (called allographs) of the
letter Q:

\begin{center}
\begin{tabular}{llr}
    \toprule
    Font                               & \TeX{} Abbreviation & Glyph \\
    \midrule
    Computer Modern (\LaTeX{} default) & \code{cmr}  & \fontfamily{cmr}\selectfont Q \\
    Latin Modern                       & \code{lmr}  & \fontfamily{lmr}\selectfont Q \\
    Johannes Kepler Roman              & \code{jkp}  & \fontfamily{jkp}\selectfont Q\\
    Johannes Kepler Roman Light        & \code{jkpl} & \fontfamily{jkpl}\selectfont Q\\
    Latin Modern Sans Serif            & \code{lmss} & \fontfamily{lmss}\selectfont Q \\
    Avant Garde                        & \code{pag}  & \fontfamily{pag}\selectfont Q \\
    Computer Modern Typewriter         & \code{cmtt} & \fontfamily{cmtt}\selectfont Q \\
    Zapf Chancery                      & \code{pzc}  & \fontfamily{pzc}\selectfont Q \\
    \bottomrule
\end{tabular}
\end{center}

In general,  font families  which one  tends to use  in a  reasonable document
these days tend to be grouped into three main categories\footnotemark:

\footnotetext{%
    There are  more, but  these are  unlikely to be  useful in  most practical
    applications these days. For more, see \cite{wikipedia:typeface}.}

\begin{titled-frame}
    {\textsf{Common Font Families}}
    \noindent{\fontfamily{ppl}\selectfont roman fonts (with serifs), for example Palatino}\\
    {\fontfamily{phv}\selectfont sans-serif fonts, for example Helvetica}\\
    {\fontfamily{pcr}\selectfont monospace fonts, for example Courier}
\end{titled-frame}

Often, font families  provide fonts for roman, sans-serif  and monospace fonts
(also called  typewriter fonts,  particularly in  the \TeX{}  world). In those
cases, the  metrics and aesthetics of  these different fonts in  the same font
family have  been tuned to look  pleasing when used together. It  is therefore
often tricky to  mix and match fonts from different  font families without the
result looking odd\footnotemark.

\footnotetext{%
    Of course it  can be done, but it  is helpful to be aware  of this general
    issue and to  be careful when picking fonts from  different families to be
    used together in the same document.}

In \LaTeX, there are three macros which contain the default fonts families for
the three font groups:

\begin{titled-frame}
    {\textsf{Default Font Families in \LaTeX}}
    \noindent\comm{rmdefault}\\
    \comm{sfdefault}\\
    \comm{ttdefault}
\end{titled-frame}

Furthermore,  there  is  a  command \comm{familydefault}  which  contains  the
currently configured default font family (one of the above three).


% -------------------------------------------------------------------------- %
\section{Selecting Fonts}
\label{sec:setting-fonts}
% -------------------------------------------------------------------------- %

There  are  various  ways  which  can  be used  to  change  fonts  at  various
places. Some of them will be described here.


% -------------------------------------------------------------------------- %
\subsection{Font Packages}
\label{subsec:font-packages}
% -------------------------------------------------------------------------- %

The easiest way to change from the  default Computer Modern to a different set
of fonts  is to use  a corresponding package. Good  places to peruse  are CTAN
\cite{ctan:fonts} and the \LaTeX{} font catalogue \cite{tug:font-catalog}.

For  example, if  we  wish to  typeset  our  document in  DejaVu,  we can  put
\compar{usepackage}{dejavu}{} in the preamble. This will set \comm{rmdefault},
\comm{sfdefault}  and  \comm{ttdefault}  to the  corresponding  values,  while
\comm{familydefault} will continue to point to the one of those three to which
it was set previously (\comm{rmdefault} by default).

Not all font packages provide all kinds of font families, or all font families
in all  weights and  shapes. Consult the documentation  for the  package which
you're  intending to  use. In  the cases  where a  package  does only  provide
some  type of  font,  the  others will  be  left  unchanged. For example,  the
\code{FiraSans} package,  which is obviously  a sans-serif typeface,  does not
provide a roman font. Therefore, the serif fonts are left untouched by loading
that package.

The same  goes for mathematics: Not all  fonts provide the needed  symbols for
typesetting that either.

A personal  favorite of mine  is the Kp-Fonts family  \cite{ctan:kpfonts}.  It
has  roman, sans-serif  and  typewriter font  choices  (although personally  I
prefer  the Computer  Modern typewriter  font and  usually keep  that intact),
along with  mathematics and  many symbols. This  is the  setup which  has been
chosen for this  document. But this is a matter of  personal preference; one's
mileage may vary.


% -------------------------------------------------------------------------- %
\subsection{Global Font Selection For a Document}
\label{subsec:fontsel:global}
% -------------------------------------------------------------------------- %

%\fontfamily{\sfdefault}\selectfont

In the preamble, we can change the  default document font from roman (which is
the default) to sans-serif or typewriter with one of these commands:

\begin{titled-frame}
{\textsf{Selecting Sans-Serif or Typewriter Default}}
\vspace{-1em}
\begin{verbatim}
\renewcommand{\familydefault}{\sfdefault}
\renewcommand{\familydefault}{\ttdefault}
\end{verbatim}
\vspace{-1em}
\end{titled-frame}

Note that typewriter fonts differ in  what kinds of adjustments \TeX{} usually
makes  when it  comes to  spacing in  order to  achieve a  justified block  of
text. This  is also  why sometimes  a monospaced  piece of  text can  protrude
outside the right textblock  margin when you put it in  normal text (see above
with the dejavu example; that was not  done on purpose). More on the topic can
be found in \cite{texblog:typewriter}.

Combining different  font families  can be  done on  the same  principle.  For
example,  selecting Palatino  as  the  roman default  font,  Helvetica as  the
default sans-serif font and Latin Modern  as the typewriter font could be done
by putting these commands in the preamble:

\begin{titled-frame}
{\textsf{Creating a Custom Set of Default Fonts}}
\vspace{-1em}
\begin{verbatim}
\renewcommand{\rmdefault}{ppl}  % Palatino
\renewcommand{\sfdefault}{phv}  % Helvetica
\renewcommand{\ttdefault}{lmtt} % Latin Modern
\end{verbatim}
\vspace{-1em}
\end{titled-frame}

Changing  math  fonts is  rather  more  complex. I  recommend not  doing  that
manually,  but  instead  relying  on   font  packages  instead,  as  described
in  the  previous  section. See for  example  \cite{stackexch:math-fonts}  and
\cite{practex:fonts}).


% -------------------------------------------------------------------------- %
\subsection{Local Font Selection Within a Document}
\label{subsec:fontsel:local}
% -------------------------------------------------------------------------- %

Besides  global  font selection,  one  might  wish  to select  fonts  manually
somewhere in a document (as opposed  to global settings in the preamble). This
is   done  by   selecting  the   desired  font family  and   then  using   the
\comm{selectfont} command. For example, if we wish to switch to Helvetica:

\begin{titled-frame}
{\textsf{Changing Fonts Until They Are Changed Again}}
\vspace{-1em}
\begin{verbatim}
\fontfamily{phv}\selectfont
\end{verbatim}


\fontfamily{phv}\selectfont  And  now  we   should  have  Helvetica  for  this
text. The font family will stay switched until it is switched back with one of
these two commands:

\begin{verbatim}
\fontfamily{\familydefault}\selectfont
\normalfont
\end{verbatim}
\normalfont And now we're back to Kp-Fonts serif.
\end{titled-frame}

In order to keep the font change local,  we can enclose a portion of text in a
group:

\begin{titled-frame}
{\textsf{Selecting Fonts Locally}}
\vspace{-1em}
\begin{verbatim}
{\fontfamily{phv}\selectfont This text is in Helvetica!}
While this text is not!
\end{verbatim}
{\fontfamily{phv}\selectfont This text is in Helvetica!} While this text is not!
\end{titled-frame}

Locally   switching    to   a   different   style    among   \comm{sfdefault},
\comm{rmdefault} and \comm{ttdefault} can be done either via:

\begin{titled-frame}
{\textsf{Locally Switching Fonts Among Default Families}}
\vspace{-1em}
\begin{verbatim}
{\rmfamily This is roman text!}
{\sffamily This is sans-serif text!}
{\ttfamily This is sans-serif text!}
\end{verbatim}
{\rmfamily This is roman text!}
{\sffamily This is sans-serif text!}
{\ttfamily This is typewriter text!}
\end{titled-frame}

Or alternatively:

\begin{titled-frame}
{\textsf{Locally Switching Fonts Among Default Families (Preferred)}}
\vspace{-1em}
\begin{verbatim}
\textrm{This is roman text!}
\textsf{This is sans-serif text!}
\texttt{This is typewriter text!}
\end{verbatim}
\vspace{-1em}
\end{titled-frame}

Leaving out the braces in the first  version results in the font being changed
until it  is switched to  something else, as  the astute reader  might already
have guessed. The second version is usually recommended for local changes.


% -------------------------------------------------------------------------- %
\section{Font Properties}
\label{sec:font-props}
% -------------------------------------------------------------------------- %

We  will  now look  at  changing  various  properties  for a  font: Its  size,
its  series (weight)  and  its  shape. For a  nice  explanation  of this,  see
\cite{stackexch:bfseries-textbf}.


% -------------------------------------------------------------------------- %
\subsection{Font Sizes}
\label{subsec:font-sizes}
% -------------------------------------------------------------------------- %

By  default, \LaTeX{}  offers the  following font  sizes, which,  depending on
the  global  document  font  size  set in  the  preamble  (for  example,  with
\code{\textbackslash{}documentclass[11pt]\{article\}}), vary in the size which
they will have on the page:

\begin{center}
    \begin{tabular}{lrrrrrr}
        \toprule
        \textsc{Size} &
        \multicolumn{3}{l}{\textsc{Standard Classes and beamer}} &
        \multicolumn{3}{l}{\textsc{AMS Classes and memoir}} \\
        \midrule
        \comm{tiny}         &  5    &  6     & 6     & 6     & 7     & 8     \\
        \comm{scriptsize}   &  7    &  8     & 8     & 7     & 8     & 9     \\
        \comm{footnotesize} &  8    &  9     & 10    & 8     & 9     & 10    \\
        \comm{small}        &  9    &  10    & 10.95 & 9     & 10    & 10.95 \\
        \comm{normalsize}   & 10    &  10.95 & 12    & 10    & 10.95 & 12    \\
        \comm{large}        & 12    &  12    & 14.4  & 10.95 & 12    & 14.4  \\
        \comm{Large}        & 14.4  &  14.4  & 17.28 & 12    & 14.4  & 17.28 \\
        \comm{LARGE}        & 17.28 &  17.28 & 20.74 & 14.4  & 17.28 & 20.74 \\
        \comm{huge}         & 20.74 &  20.74 & 24.88 & 17.28 & 20.74 & 24.88 \\
        \comm{Huge}         & 24.88 &  24.88 & 24.88 & 20.74 & 24.88 & 24.88 \\
        \bottomrule
    \end{tabular}
\end{center}

The table has been copied from \cite{wikibooks:fonts}.

Yet again, these commands change the font size within the current scope, so should
usually be enclosed in braces:

\begin{titled-frame}
{\textsf{Locally Changing Font Sizes}}
\vspace{-1em}
\begin{verbatim}
{\LARGE This is a piece of large text.} And this is of normal size again.
\end{verbatim}
{\LARGE This is a piece of large text.} And this is of normal size again.
\end{titled-frame}

% -------------------------------------------------------------------------- %
\subsection{Font Series}
\label{subsec:font-series}
% -------------------------------------------------------------------------- %

Changing font weights can be done either via a \comm{fontseries\{<argument>\}}
command,    followed   by    \comm{selectfont},   or    via   \comm{bfseries},
\comm{mdseries} and \comm{lfseries}, respectively. A  font may offer more than
three weights, in which case one might need to resort to the \comm{fontseries}
command for changing fonts.

Furthermore,  depending on  which option  is  loaded if  one is  using a  font
package  such as  Kp-Fonts, the  default font  weight and  all others  will be
changed accordingly  among \comm{bfseries} and its  friends. For example, this
happens when  one selects  the \comm{light}  option for  Kp-Fonts package. For
specifics, consult the documentation for the package you're using.

Lastly,  note that  not  all options  might  be supported  by  all fonts  (for
example,  Kp-Fonts  with  a  \code{light} option  shows  vehement  displeasure
at  being  instructed  to   execute  \comm{lfseries}). With  Kp-Fonts  in  the
configuration for this document, these are the options available:

\begin{titled-frame}
{\textsf{Locally Changing Font Weight}}
\vspace{-1em}
\begin{verbatim}
{\fontseries{l}\selectfont  The quick brown fox jumps over the lazy dog.}
{\fontseries{m}\selectfont  The quick brown fox jumps over the lazy dog.}
{\fontseries{sb}\selectfont The quick brown fox jumps over the lazy dog.}
{\fontseries{b}\selectfont  The quick brown fox jumps over the lazy dog.}
{\fontseries{bx}\selectfont The quick brown fox jumps over the lazy dog.}

{\mdseries\selectfont The quick brown fox jumps over the lazy dog.}\\
{\bfseries\selectfont The quick brown fox jumps over the lazy dog.}\\

\textbf{The quick brown fox jumps over the lazy dog.}
\textmd{The quick brown fox jumps over the lazy dog.}
\end{verbatim}

\noindent{\fontseries{l}\selectfont  The quick brown fox jumps over the lazy dog.}\\
{\fontseries{m}\selectfont  The quick brown fox jumps over the lazy dog.}\\
{\fontseries{sb}\selectfont  The quick brown fox jumps over the lazy dog.}\\
{\fontseries{b}\selectfont  The quick brown fox jumps over the lazy dog.}\\
{\fontseries{bx}\selectfont The quick brown fox jumps over the lazy dog.}\\

\noindent{\mdseries\selectfont The quick brown fox jumps over the lazy dog.}\\
{\bfseries\selectfont The quick brown fox jumps over the lazy dog.}\\

\noindent\textmd{The quick brown fox jumps over the lazy dog.}\\
\textbf{The quick brown fox jumps over the lazy dog.}\\
\vspace{-1em}
\end{titled-frame}


% TODO: Check kepler sty file for jkplight or whatever it is called


% -------------------------------------------------------------------------- %
\subsection{Font Shapes}
\label{subsec:font-shapes}
% -------------------------------------------------------------------------- %

Again, the number of font shapes, and possible combinations thereof, will vary
depending on which font is being used. Kp-Fonts supports these options:

\begin{titled-frame}
{\textsf{Locally Changing Font Shapes}}
\vspace{-1em}
\small
\begin{verbatim}
{\fontshape{sc}\selectfont The quick brown fox jumps over the lazy dog.}
{\fontshape{scsl}\selectfont The quick brown fox jumps over the lazy dog.}
{\fontshape{it}\selectfont The quick brown fox jumps over the lazy dog.}
{\otherscshape\selectfont The quick brown fox jumps over the lazy dog.}
{\otherscslshape\selectfont The quick brown fox jumps over the lazy dog.}
\end{verbatim}
\normalsize
\noindent{\fontshape{sc}\selectfont The quick brown fox jumps over the lazy dog.}\\
{\fontshape{scsl}\selectfont The quick brown fox jumps over the lazy dog.}\\
{\fontshape{it}\selectfont The quick brown fox jumps over the lazy dog.}\\
{\otherscshape\selectfont The quick brown fox jumps over the lazy dog.}\\
{\otherscslshape\selectfont The quick brown fox jumps over the lazy dog.}\\
\vspace{-1em}
\end{titled-frame}


% -------------------------------------------------------------------------- %
\subsection{Combinations}
\label{subsec:combinations}
% -------------------------------------------------------------------------- %

The  commands presented  above can  be  combined in  various ways,  depending,
again,  on the  font families  being used. Some  fonts will  support different
combinations than others.

\begin{titled-frame}
{\textsf{Combining Different Font Styles}}
\vspace{-1em}
\begin{verbatim}
The quick brown fox jumps over the lazy dog.\\
{\fontshape{sc}\fontseries{b}\selectfont
The quick brown fox jumps over the lazy dog.}\\
{\fontshape{it}\fontseries{b}\selectfont
The quick brown fox jumps over the lazy dog.}
\end{verbatim}

\noindent The quick brown fox jumps over the lazy dog.\\
{\fontshape{sc}\fontseries{b}\selectfont
The quick brown fox jumps over the lazy dog.}\\
{\fontshape{it}\fontseries{b}\selectfont
The quick brown fox jumps over the lazy dog.}
\end{titled-frame}

% -------------------------------------------------------------------------- %
\subsection{Emphasis}
% -------------------------------------------------------------------------- %

The  \comm{emph}  command  is  a   context-aware  way  to  highlight  text. It
will  select  an  appropriate  font  version  automatically  to  achieve  this
goal. Usually, \comm{emph} uses  italics, but when it is  placed inside italic
text, it will switch to an upright font shape.

\begin{titled-frame}
{\textsf{Emphasis}}
\vspace{-1em}
\small
\begin{verbatim}
{\fontshape{u}\selectfont This sentence \emph{emphasizes} a word of itself.}\\
{\fontshape{it}\selectfont This sentence \emph{emphasizes} a word of itself.}\\
Here, again, there exists an {\em alternative} version.\\
{\itshape Here, again, there exists an {\em alternative} version.}
\end{verbatim}
\normalsize
{\fontshape{u}\selectfont This sentence \emph{emphasizes} a word of itself.}\\
{\fontshape{it}\selectfont This sentence \emph{emphasizes} a word of itself.}\\
Here, again, there exists an {\em alternative} version.\\
{\itshape Here, again, there exists an {\em alternative} version.}
\end{titled-frame}

%% -------------------------------------------------------------------------- %
%\section{List of Typefaces}
%\label{sec:tflist}
%% -------------------------------------------------------------------------- %
%
%\begin{itemize}
%    \item 
%        {\fontfamily{cmr}\selectfont Computer Modern: The \LaTeX{} default}
%    \item
%        {\fontfamily{lmr}\selectfont Latin Modern: A  modernised derivative of
%        Computer modern with more characters.}
%    \item
%        {\fontfamily{jkp}\selectfont Kp-Fonts: An alternative to the above two
%        with fully-featured  glyph sets for roman,  sans-serif, typewriter and
%        math fonts. \cite{ctan:kpfonts}}
%    \item
%        {\fontfamily{aeccr}\selectfont aecc --  Almost European Concrete Roman
%        virtual  fonts.  A  font prividing  roman, sans-serif,  and typewriter
%        fonts. \cite{ctan:aecc}}
%\end{itemize}



% -------------------------------------------------------------------------- %
\begin{thebibliography}{1}
% -------------------------------------------------------------------------- %

    \bibitem{wikipedia:typeface}
        Wikipedia,
        ``\emph{Typeface}'',
        [Online],
        \href{https://en.wikipedia.org/wiki/Typeface}
             {\nolinkurl{https://en.wikipedia.org/wiki/Typeface}},
        [Accessed: 2017-MAR-20].

    \bibitem{texblog:typewriter}
        Stefan,
        ``\emph{Full justification with typewriter font}'',
        [Online],
        \href{http://texblog.net/latex-archive/plaintex/full-justification-with-typewriter-font/}
             {\nolinkurl{http://texblog.net/latex-archive/plaintex/full-justification-with-typewriter-font/}},
        2008-MAY-08,
        [Accessed: 2017-MAR-20].

    \bibitem{ctan:fonts}
        Comprehensive \TeX{} Archive Network,
        ``\emph{Topic Font: Fonts Themselves}'',
        [Online],
        \href{http://ctan.org/topic/font}
             {\nolinkurl{http://ctan.org/topic/font}},
        [Accessed: 2017-MAR-20].

    \bibitem{ctan:kpfonts}
        The Johannes Kepler Project, Christophe Caignaert,
        ``\emph{kpfonts -- A complete set of fonts for text and mathematics}''
        [Online],
        \href{https://www.ctan.org/pkg/kpfonts}
             {nolinkurl{https://www.ctan.org/pkg/kpfonts}},
        [Accessed: 2017-MAR-22].

    \bibitem{tug:font-catalog}
        \TeX{} Users Group,
        ``\emph{The \LaTeX{} Font Catalogue}'',
        [Online],
        \href{http://www.tug.dk/FontCatalogue/}
             {\nolinkurl{http://www.tug.dk/FontCatalogue/}},
        [Accessed: 2017-MAR-20].

    \bibitem{stackexch:math-fonts}
        tex.stackexchange.com,
        ``\emph{how to select math font in document}'',
        [Online],
        \href{http://tex.stackexchange.com/questions/30049/how-to-select-math-font-in-document}
             {\nolinkurl{http://tex.stackexchange.com/questions/30049/how-to-select-math-font-in-document}},
        [Accessed: 2017-MAR-20].

    \bibitem{practex:fonts}
        Walter Schmidt,
        ``\emph{Font selection in \LaTeX{}: The most frequently asked questions}'',
        The Prac\TeX{} Journal,
        [Online],
        \href{https://www.tug.org/pracjourn/2006-1/schmidt/schmidt.pdf}
             {\nolinkurl{https://www.tug.org/pracjourn/2006-1/schmidt/schmidt.pdf}},
        2006-FEB-01,
        [Accessed: 2017-MAR-20].

    \bibitem{wright:font-commands}
        David Wright,
        ``\emph{\LaTeX{} Font Commands}'',
        [Online],
        \href{http://www.cl.cam.ac.uk/~rf10/pstex/latexcommands.htm}
             {\nolinkurl{http://www.cl.cam.ac.uk/~rf10/pstex/latexcommands.htm}},
        1998,
        [Accessed: 2017-MAR-20].

    \bibitem{stackexch:bfseries-textbf}
        Tobi,
        ``\emph{\comm{bfseries} is to \comm{textbf} as WHAT is to \comm{textsf}}'',
        [Online],
        \href{http://tex.stackexchange.com/a/139592}
             {\nolinkurl{http://tex.stackexchange.com/a/139592}},
        2013-OCT-19,
        [Accessed: 2017-MAR-21].

    \bibitem{wikibooks:fonts}
        wikibooks,
        ``\emph{\LaTeX/Fonts}'',
        [Online],
        \href{https://en.wikibooks.org/wiki/LaTeX/Fonts}
             {\nolinkurl{https://en.wikibooks.org/wiki/LaTeX/Fonts}},
        [Accessed: 2017-MAR-21].

%    \bibitem{ctan:aecc}
%        CTAN,
%        ``\emph{aecc -- Almost European Concrete Roman virtual fonts}'',
%        [Online],
%        \href{http://ctan.org/pkg/aecc}
%             {\nolinkurl{http://ctan.org/pkg/aecc}},
%        [Accessed: 2017-MAr-22].

\end{thebibliography}

\end{document}
