\documentclass[a4paper,oneside,11pt]{article}

\usepackage{lipsum}
\usepackage{booktabs}
\usepackage{hyperref}
\usepackage{hologo}
\usepackage[light,nott]{kpfonts}

\newcommand\compar[2]{\texttt{\textbackslash #1\{#2\}}}
\newcommand\comm[1]{\texttt{\textbackslash#1}}
\newcommand\code[1]{\texttt{#1}}
\newcommand\XeLaTeX{\hologo{XeLaTeX}}

\usepackage[T1]{fontenc}
\usepackage[utf8]{inputenc}

\title{Changing Fonts in \LaTeX}
\author{Raphael Frey  <\texttt{webmaster@alpenwasser.net}>}


% ************************************************************************** %
\begin{document}
% ************************************************************************** %

\maketitle

\tableofcontents

% -------------------------------------------------------------------------- %
\section{Font Families, Fonts and Glyphs}
\label{sec:font-families}
% -------------------------------------------------------------------------- %

A \emph{font family},  also referred to as  a \emph{typeface}, is a  font or a
set  of fonts  with common  features and  characteristics, comprising  all the
needed \emph{glyphs}. A glyph, in this  context, is a graphical representation
of a character,  which can be a  letter, a logogram, a  mathematical symbol or
other things\footnotemark.

\footnotetext{%
    Obviously typography is a vast and complex topic and there is no way I can
    do it  true justice  here. Even if  I wanted  to, I  am most  certainly no
    scholar  on  the  topic. But  for  the  purposes  of  this  document  this
    information shall suffice. The curious reader  shall feel free to research
    the topic in more depth, if so desired.}

For example, the following are all different glyphs (called allographs) of the
letter Q:

\begin{center}
\begin{tabular}{llr}
    \toprule
    Font                               & \TeX{} Abbreviation & Glyph \\
    \midrule
    Computer Modern (\LaTeX{} default) & \code{cmr}  & \fontfamily{cmr}\selectfont Q \\
    Latin Modern                       & \code{lmr}  & \fontfamily{lmr}\selectfont Q \\
    Johannes Kepler Roman              & \code{jkp}  & \fontfamily{jkp}\selectfont Q\\
    Johannes Kepler Roman Light        & \code{jkpl} & \fontfamily{jkpl}\selectfont Q\\
    Latin Modern Sans Serif            & \code{lmss} & \fontfamily{lmss}\selectfont Q \\
    Avant Garde                        & \code{pag}  & \fontfamily{pag}\selectfont Q \\
    Computer Modern Typewriter         & \code{cmtt} & \fontfamily{cmtt}\selectfont Q \\
    Zapf Chancery                      & \code{pzc}  & \fontfamily{pzc}\selectfont Q \\
    \bottomrule
\end{tabular}
\end{center}

In general,  font families  which one  tends to use  in a  reasonable document
these days tend to be grouped into three main categories\footnotemark:

\footnotetext{%
    There are  more, but  these are  unlikely to be  useful in  most practical
    applications these days. For more, see \cite{wikipedia:typeface}.}

\begin{itemize}
    \item {\fontfamily{ppl}\selectfont roman fonts (with serifs), for example Palatino}
    \item {\fontfamily{phv}\selectfont sans-serif fonts, for example Helvetica}
    \item {\fontfamily{pcr}\selectfont monospace fonts, for example Courier}
\end{itemize}

Often, font families  provide fonts for roman, sans-serif  and monospace fonts
(also called  typewriter fonts,  particularly in  the \TeX{}  world). In those
cases, the  metrics and aesthetics of  these different fonts in  the same font
family have  been tuned to look  pleasing when used together. It  is therefore
often tricky to  mix and match fonts from different  font families without the
result looking odd\footnotemark.

\footnotetext{%
    Of course it  can be done, but it  is helpful to be aware  of this general
    issue and to  be careful when picking fonts from  different families to be
    used together in the same document.}

In \LaTeX, there are three macros which contain the default fonts families for
the three font groups:

\begin{itemize}
    \item \comm{rmdefault}
    \item \comm{sfdefault}
    \item \comm{ttdefault}
\end{itemize}

Furthermore,  there  is  a  command \comm{familydefault}  which  contains  the
currently configured default font family (one of the above three).

% -------------------------------------------------------------------------- %
\section{Selecting Fonts}
\label{sec:setting-fonts}
% -------------------------------------------------------------------------- %

There  are  various  ways  which  can  be used  to  change  fonts  at  various
places. Some of them will be described here.


% -------------------------------------------------------------------------- %
\subsection{Font Packages}
\label{subsec:font-packages}
% -------------------------------------------------------------------------- %

The easiest way to change from the  default Computer Modern to a different set
of fonts  is to use  a corresponding package. Good  places to peruse  are CTAN
\cite{ctan:fonts} and the \LaTeX{} font catalogue \cite{tug:font-catalog}.

For  example, if  we  wish to  typeset  our  document in  DejaVu,  we can  put
\compar{usepackage}{dejavu}{} in the preamble. This will set \comm{rmdefault},
\comm{sfdefault}  and  \comm{ttdefault}  to the  corresponding  values,  while
\comm{familydefault} will continue to point to the one of those three to which
it was set previously (\comm{rmdefault} by default).

Not all font packages provide all kinds of font families, or all font families
in all  weights and  shapes. Consult the documentation  for the  package which
you're  intending to  use. In  the cases  where a  package  does only  provide
some  type of  font,  the  others will  be  left  unchanged. For example,  the
\code{FiraSans} package,  which is obviously  a sans-serif typeface,  does not
provide a roman font. Therefore, the serif fonts are left untouched by loading
that package.

The same  goes for mathematics: Not all  fonts provide the needed  symbols for
typesetting that either.

A personal favorite of mine is  the Johannes Kepler font family. It has roman,
sans-serif  and typewriter  font  choices (although  personally  I prefer  the
Computer  Modern typewriter  font and  usually keep  that intact),  along with
mathematics and many symbols. This is the setup which has been chosen for this
document. But this is a matter of personal preference; one's mileage may vary.


% -------------------------------------------------------------------------- %
\subsection{Global Font Selection For a Document}
\label{subsec:fontsel:global}
% -------------------------------------------------------------------------- %

%\fontfamily{\sfdefault}\selectfont

In the preamble, we can change the  default document font from roman (which is
the default) to sans-serif or typewriter with

\begin{verbatim}
\renewcommand{\familydefault}{\sfdefault}
\end{verbatim}
and
\begin{verbatim}
\renewcommand{\familydefault}{\ttdefault}
\end{verbatim}
respectively.

\emph{Note:} Typewriter  fonts  differ in  what  kinds  of adjustments  \TeX{}
usually make when it comes to spacing in order to achieve a justified block of
text. More on the topic can be found in \cite{texblog:typewriter}.

Combining different  font families  can be  done on  the same  principle.  For
example,  selecting Palatino  as  the  roman default  font,  Helvetica as  the
default sans-serif font and Latin Modern  as the typewriter font could be done
by putting these commands in the preamble:

\begin{verbatim}
\renewcommand{\rmdefault}{ppl}
\renewcommand{\sfdefault}{phv}
\renewcommand{\ttdefault}{lmtt}
\end{verbatim}

Changing  math  fonts is  rather  more  complex. I  recommend not  doing  this
manually,   but  instead   relying  on   packages  instead. See   for  example
\cite{stackexch:math-fonts} and \cite{practex:fonts}).


% -------------------------------------------------------------------------- %
\subsection{Local Font Selection Within a Document}
\label{subsec:fontsel:local}
% -------------------------------------------------------------------------- %

Besides  global  font selection,  one  might  wish  to select  fonts  manually
somewhere in a document (as opposed to global settings in the preamble).


% -------------------------------------------------------------------------- %
\subsection{pdf\LaTeX{} vs. \XeLaTeX{} vs. Lua\LaTeX}
\label{subsec:fontsel:pdf:xe:lua}
% -------------------------------------------------------------------------- %


% -------------------------------------------------------------------------- %
\section{List of Typefaces}
\label{sec:tflist}
% -------------------------------------------------------------------------- %

\begin{itemize}
    \item Computer Modern: The \LaTeX{} default
    \item  
        Latin  Modern: Basically  Computer  Modern with  some  amendments  for
        non-English languages.
\end{itemize}


% -------------------------------------------------------------------------- %
\begin{thebibliography}{1}
% -------------------------------------------------------------------------- %
        
    \bibitem{wikipedia:typeface}
        Wikipedia,
        \emph{Typeface},
        [Online],
        \href{https://en.wikipedia.org/wiki/Typeface}
             {\nolinkurl{https://en.wikipedia.org/wiki/Typeface}},
        [Accessed: 2017-MAR-20].

    \bibitem{texblog:typewriter}
        Stefan,
        \emph{Full justification with typewriter font},
        [Online],
        \href{http://texblog.net/latex-archive/plaintex/full-justification-with-typewriter-font/}
             {\nolinkurl{http://texblog.net/latex-archive/plaintex/full-justification-with-typewriter-font/}},
        2008-MAY-08,
        [Accessed: 2017-MAR-20].

    \bibitem{ctan:fonts}
        Comprehensive \TeX{} Archive Network,
        \emph{Topic Font: Fonts Themselves},
        [Online],
        \href{http://ctan.org/topic/font}
             {\nolinkurl{http://ctan.org/topic/font}},
        [Accessed: 2017-MAR-20].

    \bibitem{tug:font-catalog}
        \TeX{} Users Group,
        \emph{The \LaTeX{} Font Catalogue},
        [Online],
        \href{http://www.tug.dk/FontCatalogue/}
             {\nolinkurl{http://www.tug.dk/FontCatalogue/}},
        [Accessed: 2017-MAR-20].

    \bibitem{stackexch:math-fonts}
        tex.stackexchange.com,
        \emph{how to select math font in document},
        [Online],
        \href{http://tex.stackexchange.com/questions/30049/how-to-select-math-font-in-document}
             {\nolinkurl{http://tex.stackexchange.com/questions/30049/how-to-select-math-font-in-document}},
        [Accessed: 2017-MAR-20].
        
    \bibitem{practex:fonts}
        Walter Schmidt,
        \emph{Font selection in \LaTeX{}: The most frequently asked questions},
        The Prac\TeX{} Journal,
        [Online],
        \href{https://www.tug.org/pracjourn/2006-1/schmidt/schmidt.pdf}
             {\nolinkurl{https://www.tug.org/pracjourn/2006-1/schmidt/schmidt.pdf}},
        2006-FEB-01,
        [Accessed: 2017-MAR-20].


\end{thebibliography}

\end{document}
