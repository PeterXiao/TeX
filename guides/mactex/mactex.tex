\documentclass[a4paper,11pt]{article}
\usepackage[left=30mm,right=30mm,top=30mm,bottom=30mm]{geometry}
\usepackage[ngerman]{babel}
\usepackage{kpfonts}
\usepackage{graphicx}
\usepackage{hyperref}
\renewcommand{\familydefault}{\sfdefault}
\def\mt{Mac\TeX}
\newcommand{\screenshot}[1]{%
    \noindent\begin{minipage}{\textwidth}
        \vspace{1em}
        \centering
        \includegraphics[width=0.8\textwidth]{images/#1}
        \vspace{1em}
    \end{minipage}}
\title{\mt{} Installation Guide}
\author{Raphael Frey}
\date{\today}

\begin{document}
\maketitle

Der Installer kann heruntergeladen werden von
\href{https://www.tug.org/mactex/}{https://www.tug.org/mactex/}
(\emph{MacTeX Download}).


%\screenshot{mactex00.png}

Ausser man ist direkt an einem Internet-Backbone, wird dies eine Weile dauern.

\screenshot{mactex01.png}

Anschliessend  den   Installer  ausf\"uhren  und   durchklicken. Sollte  alles
ziemlich selbsterkl\"arend sein.

\screenshot{mactex02.png}
\screenshot{mactex03.png}
\screenshot{mactex04.png}
\screenshot{mactex05.png}
\screenshot{mactex06.png}
\screenshot{mactex07.png}
\screenshot{mactex08.png}

\newpage
Allenfalls  m\"ussen noch  die  Entwicklertools  installiert werden. Hat  beim
Autor alles reibungslos funktioniert. Immer sch\"on weiterklicken.

\screenshot{mactex09.png}
\screenshot{mactex10.png}
\screenshot{mactex11.png}
\screenshot{mactex12.png}

\newpage
Und voil\`a!

\screenshot{mactex13.png}

Unter  anderem beinhaltet  \mt{}  ein Administrations-Tool,  mit dem  Packages
installiert, deinstalliert und aktualisiert werden k\"onnen. Im Folgenden wird
ein Update-Durchlauf gezeigt.

\vspace{1em}
Beim Starten wird man allenfalls von dieser Sicherheitsmeldung begr\"usst.

\screenshot{mactex14.png}

\newpage
Updates verf\"ugbar!

\screenshot{mactex15.png}

Zuerst wird  die Datenbank mit dem  Server synchronisiert und mit  der lokalen
Installation verglichen. Ist  dieser Vorgang erfolgreich, sieht  das wie folgt
aus:

\screenshot{mactex16.png}

\newpage
Anschliessend den Befehl erteilen, alle Packages zu aktualisieren.

\screenshot{mactex17.png}

Durchklicken\ldots

\screenshot{mactex18.png}
%\screenshot{mactex19.png}
%\screenshot{mactex20.png}

\newpage
Das geht dann allenfalls eine Weile.

\screenshot{mactex21.png}

\vspace{1em}
Wenn alles nach Plan gelaufen ist, sollte man hier landen:

\screenshot{mactex22.png}

%\screenshot{mactex23.png}

\end{document}
