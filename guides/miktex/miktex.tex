\documentclass[a4paper,11pt]{article}
\usepackage[left=30mm,right=30mm,top=30mm,bottom=30mm]{geometry}
\usepackage[ngerman]{babel}
\usepackage{kpfonts}
\usepackage{graphicx}
\usepackage{hyperref}
\renewcommand{\familydefault}{\sfdefault}
\def\mt{MiKTeX}
\newcommand{\screenshot}[1]{%
    \begin{minipage}{\textwidth}
        \vspace{1em}
        \centering
        \includegraphics[width=0.8\textwidth]{images/#1}
        \vspace{1em}
    \end{minipage}}
\title{\mt~Installation Guide}
\author{Raphael Frey}
\date{\today}

\begin{document}
\maketitle

% ---------------------------------------------------------------------------- %
\section{Download}
\label{sec:download}
% ---------------------------------------------------------------------------- %

Die Installer k\"onnen heruntergeladen werden von: \href{https://miktex.org/download}{https://miktex.org/download}

\mt~kann dynamisch Packages installieren,  welche f\"ur ein Projekt ben\"otigt
werden  und noch  nicht auf  dem System  installiert sind. Es  ist also  nicht
unbedingt notwendig, zu Beginn den  gesamten ``Garten zu pflanzen''. In diesem
Falle w\"ahle man auf der \mt-Website die Option \emph{Masic \mt~Installer} in
der ben\"otigten Bit-Version (normalerweise 64 dieser Tage).

Andererseits  kann  es  damit  nat\"urlich leicht  geschehen,  dass  man  beim
Arbeiten  an einem  Dokument noch  auf  das Installieren  von Packages  warten
muss;  ebenfalls  wird in  diesem  Falle  logischerweise eine  funktionierende
Internet-Verbindung ben\"otigt, um das fehlende Package noch zu installieren.

Zur  Installation der  gesamten Distribution  w\"ahle man  den \emph{\mt~  Net
Installer}. Dies ist die Variante, die in dieser Anleitung befolgt wird.

Schlussendlich  ist es  aber eine  Frage der  pers\"onlichen Vorliebe. 

Nach  dem Download  f\"uhre man  den Installer  aus, und  folge den  Schritten
gem\"ass den Screenshots.

Der \"ubliche Lizenzbrei, den wir sicher alle brav lesen und verstehen:

\screenshot{miktex00.png}

Die Installation besteht aus  zwei Schritten: Zuerst werden alle gew\"unschten
Pakete in  ein Verzeichnis  heruntergeladen (obere Option),  und anschliessend
wird \mt~ aus diesem installiert (untere Option). Zumindest bei meinem Versuch
resultierte die untere Option nicht im automatischen Ausf\"uhren der ersten.

Also: Zuerst mal runterladen.

\screenshot{miktex01.png}

Auch hier kann man nochmals w\"ahlen, ob man eine Gesamtinstallation oder eine
teilweise Installation m\"ochte.

\screenshot{miktex02.png}

Einen Server ausw\"ahlen, von dem die Daten heruntergeladen werden sollen:

\screenshot{miktex03.png}

\screenshot{miktex04.png}

\newpage
Lokales Zielverzeichnis:

\screenshot{miktex05.png}

Blablabla:

\screenshot{miktex06.png}

\newpage
Und Kaffee trinken gehen\ldots

\screenshot{miktex07.png}

Nach Beendigung des Download-Prozesses, ist es nicht auszuschliessen, dass der
Installer sich beim Klicken anf \emph{Next} gem\"ass untenstehendem Screenshot
beended\footnotemark. In  dem  Falle  nochmals  ausf\"uhren  und  durchklicken
zu\ldots
\footnotetext{Zumindest war dies die Erfahrung es Autors.}

\screenshot{miktex08.png}

\newpage
\ldots  diesem  Dialog. Jetzt also  die  untere  Option ausw\"ahlen,  und  das
Verzeichnis angeben, wohin vorher die Daten gespeichert worden sind.

\screenshot{miktex09.png}

Ein paar Mal durchklicken:

\screenshot{miktex10.png}

\screenshot{miktex11.png}

\screenshot{miktex12.png}

\screenshot{miktex13.png}

Falls  man  zu  den  armen   Seelen  geh\"ort,  die  h\"aufig  nicht-metrische
Papierformate verwenden m\"ussen, kann  man hier noch die Standard-Einstellung
\"andern:

\screenshot{miktex14.png}

\screenshot{miktex15.png}

Mehr Kaffee trinken\ldots

\screenshot{miktex16.png}

\screenshot{miktex17.png}

\screenshot{miktex18.png}

Und fertig!

F\"ur  diejenigen,  welche  ihre  neue  \mt-Installation  gleich  ausprobieren
wollen: Der  standardm\"assig  installierte   Editor  ist  TeXWorks. Ebenfalls
sind  noch einige  Administrations-Tools  installiert worden,  um Packages  zu
aktualisieren etc. Sollte alles ziemlich selbsterkl\"arend sein.

\end{document}
