\documentclass{article}
\usepackage{lipsum}    % dummy text
% -------------------------------------------------------- %
% Provides  a   new  environment  a3pages   for  inserting %
% landscape a3pages at arbitrary points in your document.  %
%                                                          %
% NOTE: This environment needs to be enclosed in braces to %
% work  correctly.  Unfortunately,  I  haven't managed  to %
% integrate  those into  the commands  directly yet  (they %
% obviously mess up the bracing of the command itself).    %
%                                                          %
% EXAMPLE:                                                 %
%                                                          %
% {\begin{a3pages}                                         %
%     your content                                         %
%     goes here                                            %
%     and will be put on                                   %
%     landscape A3 pages                                   %
% \end{a3pages}}                                           %
%                                                          %
% Note the  opening brace  before \begin{a3pages}  and the %
% additional closing brace after \end{a3pages}             %
% -------------------------------------------------------- %
% See also:
% http://tex.stackexchange.com/questions/16942/difference-between-textwidth-linewidth-and-hsize

% Memoir class version:
% NOTE: Relies on the spefic margins as set in my memoir.tex file
\newenvironment{a3pagesMem}
    {%
        \clearpage
        \setlength{\pdfpagewidth}{2\pdfpagewidth}
        \setlength{\hsize}{\pdfpagewidth} % for text paragraphs
        \addtolength{\hsize}{-\spinemargin}
        \addtolength{\hsize}{-\foremargin}
        \setlength{\textwidth}{\hsize}                             % headers, footers
        \setlength{\stockwidth}{2\stockwidth}
        \setlength{\paperwidth}{2\paperwidth}
        \checkandfixthelayout%
    }
    {%
        \clearpage
    }

% Version for Non-memoir classes (article etc.)
% https://en.wikibooks.org/wiki/LaTeX/Page_Layout
\newenvironment{a3pages}
    {%
        \clearpage
        \setlength{\pdfpagewidth}{2\pdfpagewidth}
        \setlength{\hsize}{\pdfpagewidth}
        \addtolength{\hsize}{-2\oddsidemargin}
        \addtolength{\hsize}{-2\marginparwidth}
        \addtolength{\hsize}{-2\marginparsep}
        \setlength{\textwidth}{\hsize}
        \setlength{\paperwidth}{2\paperwidth}
    }
    {%
        \clearpage
    }


% ------------------------------------------------ %
% Picture example source:                          %
% https://en.wikibooks.org/wiki/LaTeX/Picture      %
% ------------------------------------------------ %
\def\testpicture{%
    \setlength{\unitlength}{2cm}
    \noindent\begin{picture}(6,5)
        \thicklines
        \put(1,0.5){\line(2,1){3}}
        \put(4,2){\line(-2,1){2}}
        \put(2,3){\line(-2,-5){1}}
        \put(0.7,0.3){$A$}
        \put(4.05,1.9){$B$}
        \put(1.7,2.95){$C$}
        \put(3.1,2.5){$a$}
        \put(1.3,1.7){$b$}
        \put(2.5,1.05){$c$}
        \put(0.3,4){$F=\sqrt{s(s-a)(s-b)(s-c)}$}
        \put(3.5,0.4){$\displaystyle s:=\frac{a+b+c}{2}$}
    \end{picture}}

\begin{document}
\lipsum[1-3]


{\begin{a3pages} % The opening brace is needed (see also a3pages.tex)
    % ---------------------------------------------------- %
    % Fill up an A3 page  with lots and lots of text. This %
    % will also flow caross several pages correctly.       %
    % ---------------------------------------------------- %
    \lipsum[4-27]

    % ---------------------------------------------------- %
    % The  \newpage prevents  the paragraphs  on the  last %
    % page before  the page with the  minipages from being %
    % spread over the entire  page vertically, which would %
    % look terrible.                                       %
    % ---------------------------------------------------- %
    \newpage

    % ---------------------------------------------------- %
    % NOTE: Minipages do  not break across  pages. If they %
    % are  filled  up with  too  much  content, they  will %
    % simply flow off the page.                            %
    %                                                      %
    % Because minipages are treated like any other element %
    % in the text,  they would be indented  by default (as %
    % the  first element  in their  respective paragraph), %
    % throwing  off  the  horizontal spacing.   Hence  the %
    % \noindent commands.                                  %
    % ---------------------------------------------------- %
    \noindent\begin{minipage}[t]{0.45\textwidth} % top-aligned minipages
        \lipsum[18-25]
    \end{minipage}\hspace*{0.1\textwidth}
    \noindent\begin{minipage}[t]{0.45\textwidth} % top-aligned minipages
        \vspace{-5em} % Yes, this is hacky...
        \begin{center}
            \testpicture
        \end{center}

        \lipsum[26-29]
    \end{minipage}

    \noindent\begin{minipage}[b]{0.45\textwidth} % bottom-aligned minipages
        \lipsum[18-25]
    \end{minipage}\hspace*{0.1\textwidth}
    \noindent\begin{minipage}[b]{0.45\textwidth} % bottom-aligned minipages
        \begin{center}
            \testpicture
        \end{center}

        \lipsum[26-28]
    \end{minipage}
\end{a3pages}} % Mind the closing brace

% Inserting some random A4 pages, because why not.
\lipsum[20-24]

{\begin{a3pages}

    \noindent\begin{minipage}[t]{0.45\textwidth}
        \lipsum[18-23]
    \end{minipage}\hspace*{0.1\textwidth}
    \noindent\begin{minipage}[t]{0.45\textwidth}
        \lipsum[26-28]
    \end{minipage}

\end{a3pages}} % Mind the closing brace
\lipsum[10-12]
\end{document}
